\documentclass[12pt]{article}

% set margins and spacing
\addtolength{\textwidth}{1.3in}
\addtolength{\oddsidemargin}{-.65in} %left margin
\addtolength{\evensidemargin}{-.65in}
\setlength{\textheight}{9in}
\setlength{\topmargin}{-.5in}
\setlength{\headheight}{0.0in}
\setlength{\footskip}{.375in}
\renewcommand{\baselinestretch}{1.0}
\linespread{1.0}

% load miscellaneous packages
\usepackage{csquotes}
\usepackage[american]{babel}
\usepackage[usenames,dvipsnames]{color}
\usepackage{graphicx,amsbsy,amssymb, amsmath, amsthm, MnSymbol,bbding,times, verbatim,bm,pifont,pdfsync,setspace,natbib}

% enable hyperlinks and table of contents
\usepackage[pdftex,
bookmarks=true,
bookmarksnumbered=false,
pdfview=fitH,
bookmarksopen=true,hyperfootnotes=false]{hyperref}

% define environments
\newtheorem{definition}{Definition}
\newtheorem{fact}{Fact}
\newtheorem{result}{Result}
\newtheorem{proposition}{Proposition}



\begin{document}
\title{What are the impacts of US trade policy changes on trade with the European Union?}
\author{Arya Rajesh\thanks{arajesh@syr.edu.} \and Sawyer Quinn McFarland\thanks{sqmcfarl@syr.edu} \and Logan Riley Faulk\thanks{lrfaulk@syr.edu}}
\date{\vskip-.1in \today}
\maketitle 

\vskip.3in
\begin{center} {\bf Abstract} \end{center}

\begin{quote}
{\small Sawyer McFarland, Logan Faulk, and Arya R investigate the repercussions of US trade policy changes on EU exports to the United States, focusing on the 2018 tariffs imposed on Chinese exports. Amidst escalating US-China trade tensions, the study hypothesizes that the 2019 US tariffs on Chinese goods reduced demand, subsequently boosting the demand for EU-exported equivalents to the United States. The research integrates insights from a thorough literature review, leveraging seminal papers on the effects of US trade policy changes on EU trade relations. Theoretical foundations such as competitive interdependence, comparative advantage, and the J-Curve phenomenon provide a framework for understanding economic intricacies. Utilizing primary data sets from WITS and Eurostat, the study concentrates on textiles and apparel, food, and electrical equipment. Comparative analysis of US imports from China and the EU pre, during, and post-tariff implementation reveals a consistent increase in EU exports to the US across all three categories. Despite temporary COVID-19 impacts in 2020, the subsequent rebound in 2021 supports the hypothesized trajectory, showcasing the resilience of identified trends. However, limitations in data granularity highlight potential avenues for future research improvement. The findings underscore the tangible impact of US tariffs on Chinese exports, positioning the EU as a beneficiary in key product groups, contributing to the ongoing discourse on the complex consequences of trade policy decisions in our globally interconnected economy.}
\end{quote}

\bigskip
\section{Introduction} \label{sec:introduction}

Understanding the far-reaching consequences of US trade policy decisions on global dynamics is important for policymakers, economists, and corporations. This research seeks to address the tangible impacts of these decisions, especially amid escalating tensions with China, providing valuable insights into international commerce. The choice of this topic stems from a genuine interest in unraveling the real-world implications of US trade policy, contributing significantly to a comprehensive understanding of global economic dynamics. The central question we tackle is the impact of US tariffs on Chinese exports in 2019 on EU exports to the United States. To answer this question, we conduct a meticulous examination of trade data using primary datasets like WITS and Eurostat. Our focus on three key product groups—food, textiles, and capital goods—offers a nuanced understanding of the interconnected effects of tariffs on international commerce.
Despite facing challenges such as the COVID-19 pandemic and data limitations, our findings reveal a tangible impact on EU-US trade dynamics. Key product groups show the EU emerging as a beneficiary, supporting our hypothesis that US tariffs on Chinese goods influenced the demand for EU-exported equivalents to the United States. The paper unfolds seamlessly, beginning with an introduction that sets the context and introduces the central research question. It then transitions into a detailed methodology, outlining the approach taken and emphasizing the use of primary data sets. Subsequent sections present empirical findings, focusing on trends in the selected product groups. A comprehensive economic analysis follows, exploring the broader implications of the research. The paper concludes with acknowledgments of limitations and proposals for future studies, providing a cohesive journey through theoretical foundations, empirical analyses, and practical implications of the research.\end{enumerate}

\section{Literature Review} \label{sec:literature}

The ever-shifting landscape of global trade is a complex tapestry woven with intricate interdependencies, competitive strategies, and policy decisions reverberating across economies. To delve into this dynamic realm, a comprehensive literature review has been conducted, exploring five seminal papers illuminating the impacts of US trade policy changes on trade with the European Union (EU). This exploration spans theories such as competitive interdependence, comparative advantage, and the J-Curve phenomenon, offering a nuanced understanding of the economic intricacies. Sbragia's work in 2010 on "Competitive Interdependence in Globalization Management" lays a foundational framework for comprehending the effects of trade policy choices on the EU's economic outcomes in its trade relations with the US and China. Intricately discussed in the paper, the concept of competitive interdependence underscores the paradoxical nature of competition and cooperation in the global economy. This aligns seamlessly with economic principles discussed by Remler \& Van Ryzin in Chapter 2, emphasizing the influence of trade policy choices on creating competitive interdependence. This theory sets the stage for analyzing changes in trade policy and their subsequent impacts on traded goods, trade volumes, pricing dynamics, and the timing of transactions between the EU and the US.
Demertzis and Fredriksson's 2018 contribution, "Comparative Advantage and Economic Welfare Analysis," offers a lens through which to examine the EU's response to US trade tariffs. By leveraging the Theory of Comparative Advantage, Trade Multiplier Effect, and Economic Welfare Analysis, the study emphasizes the importance of economic welfare considerations, aligning directly with economic principles discussed in Remler \& Van Ryzin Chapter 2. The paper substantiates the hypothesis that tariffs disrupt the normal flow of international trade and trigger adverse economic consequences, providing a crucial link between theoretical economic principles and real-world policy implications.
Fajgelbaum et al. (2021): The US-China Trade War and Global Re allocations introduces a broader perspective by investigating the impacts of the US-China trade war on global re allocations. While not directly focused on the EU, the research enriches our understanding by considering bystander countries worldwide, highlighting the interconnections of global trade networks and potential ramifications for the EU. This expanded scope broadens the context within which US trade policy changes affect international trade dynamics.
"Balistreri et al. (2018): Quantifying Disruptive Trade Policies" focuses on the European response to US trade tariffs, providing a detailed analysis of economic impacts. The paper's relevance lies in examining how European nations respond to economic sanctions resulting from the US-China trade war. Grounded in empirical assessments, the study contributes valuable insights into the intricate web of consequences triggered by trade policy changes, particularly as European countries navigate disruptions and seek economic benefits through trade diversion.
Adding a historical dimension to the discussion, Bahmani-Oskooee's 1985 exploration of the J-Curve phenomenon, titled "Devaluation and the J-Curve," offers insights into the consequences of currency devaluation on trade balances. The hypothesis challenges conventional wisdom, suggesting that countries with a higher devaluation rate may experience worse trade balances. This historical perspective prompts a reconsideration of the effectiveness of devaluing currency to alleviate trade deficits, especially in modern trade policy with larger nations carrying high levels of debt, such as the United States. By weaving together insights from these diverse papers, this synthesis provides a multi-dimensional understanding of international trade's complex dynamics. The amalgamation of competitive interdependence, comparative advantage, global re allocations, and currency devaluation offers a solid foundation for policymakers, economists, and researchers striving to comprehend and navigate the evolving landscape of global commerce. These insights serve as a compass to navigate the intricacies of contemporary trade policy, fostering a comprehensive approach to decision-making in an ever-evolving global economic landscape.





\section{Data}
\label{sec:data}



In this report, the data utilized was provided by Professor Buzzard and Professor Khan, comprising two Excel datasets. The first set encompasses trade item category, trade volume, money, year, and other variables relevant to trade between the United States and China. The second set includes specific products traded, trade volume in Euros and product scale, transaction period, and whether the item was imported or exported among all European Union countries and the United States.

To generate visualizations, specific variables or products were employed when establishing rules for the code. The process for creating histograms for all eight variables remains consistent across all graphs. Minor modifications, such as altering the traded item's name, are the only necessary adjustments. The sequential steps I followed are as follows:

\begin{itemize}
    \item Open R Studio and create a new project and R script.
    \item Open the Excel files used and save them to a cloud folder.
    \item Set the working directory to open Excel files using the set working directory function under the session on the menu bar. Choose the cloud folder where the Excel files are stored and insert the desired Excel file name into the function.
    \item Use read.csv(‘’) to read the data and name the function df - read.csv(‘excel file.csv’).
    \item Install necessary packages (ggplot2, dplyr, tidyverse) using the install.packages(“”) function.
    \item Filter down the data frames used and name the filtered data frame as filtered df.
    \item Use filter() to select specific columns for the new data set: filtered df - df filter(Reporter.Name == ”United States”, Partner.Name == ”China”, Product.Group == ” Capital goods”).
    \item Repeat the above steps for all four variables used in the U.S. to China data set (capital goods, food, textiles, chemicals).
    \item Create a new data set for the desired year range, named time range df: time range df - filtered df Select(x2017:x2021).
    \item Modify the year names to make the data suitable for plotting: long df2 - pivot longer(time range df2, cols = X2017:X2021, names to = ”Year”, values to = ”Value”) and long df2Year ¡- sub(”x”, ””, long dfYear).
\end{itemize}
Now, proceed to create graphs using ggplot for all eight variables. The following steps should be repeated for both the U.S. and China data and the U.S. and EU data:

\begin{itemize}
    \item Use ggplot() with aes(x = Year, y = Value, group = 1) to set the x-axis and y-axis variables.
    \item Add geom line() to create a line graph.
    \item Add geom point() to plot distinct points on the graph.
    \item Use labs(title = ”Food Trade U.S. and China (2017-2021)”, x = ”Year”, y = ”Trade Volume”) to set titles for the graph.
    \item Apply theme minimal() to remove most of the background and grid lines.
\end{itemize}
By following these steps, a total of eight histograms, four from the United States and China trade data and four from the EU and United States trade data were generated, each representing the chosen categories (chemicals, food, textiles, capital goods).



\section{Results}
\label{sec:result}
The research commenced by meticulously collecting and analyzing data on specific product categories' US imports between 2016 and 2020. This exploration, carried out using WITS and Eurostat datasets, honed in on key categories—textiles and apparel (HS 50–63), food (HS 84 and 90), and electrical equipment (HS 85)—recognized for China's competitive exports. The primary focus was to dissect the decline in US imports of these products from China during the critical period of 2018 to 2019, seeking to unravel the intricate dynamics influencing EU-US trade. In the assessment of the impact, a nuanced approach was taken, assuming that 63 percent of the reduced demand for these products in the US market in 2019 redirected to other countries. This assumption, grounded in insights from WITS (World Integrated Trade Solution), provided a lens through which the subsequent steps unfolded. The meticulous calculation of the total trade redirected from China to third-party countries unveiled distinct beneficiaries among the EU 27 countries. Notably, the study identified specific countries that capitalized on the shift in trade dynamics. Italy emerged as a notable beneficiary in textiles, showcasing the resilience and adaptability of its export sector. In the realm of food, Switzerland and France stood out, leveraging the changes in trade patterns to enhance their positions in the US market. The machinery sector witnessed a significant boost for Germany, further solidifying its role as an economic powerhouse.

Both WITS and Eurostat datasets concurred, portraying a conspicuous decrease in US imports of food, capital goods, and textiles from China post-2018. This downturn reflected the tangible impact of tariffs on Chinese goods. In stark contrast, the US-EU dataset revealed a corresponding increase in exports of these categories, aligning seamlessly with the overarching hypothesis of the study. Examining the broader context, the US trade deficit with China experienced a modest reduction in 2019, reaching \$342 billion, though remaining comparable to the 2016 level of \$346 billion. However, the comprehensive US trade deficit with the world surged to \$845 billion in 2019 from \$735 billion in 2016, underlining the far-reaching consequences of the tariffs on the US economy. Acknowledging the resilience of Chinese exports to the United States, with 75 percent persevering through the administration's tariffs and being purchased in the US, the study underscored the strategic positioning of the European Union. Leveraging insights from both WITS and Eurostat, the EU, through specific countries like Italy, Switzerland, France, and Germany, emerged as the primary beneficiary of the US-China tariff war. This detailed exploration, supported by robust datasets, provided a comprehensive understanding of the multifaceted impacts on global trade dynamics. It is crucial to note that the assessment's temporal scope extends up to the first half of 2019, excluding the effects of additional Section 301 tariffs imposed by the Trump administration in June and September 2019. We had to acknowledge that certain products in the US-China dataset maintained stability due to China's clear comparative advantage in labor-intensive goods and the US's advantage in high-tech machinery. This strategic consideration elucidates the intricate balance in global trade dynamics. 


\section{Discussion}
\label{sec:discussion}
While we collected our data and developed graphs for our product groups, we discovered a significant caveat: COVID-19. In all three product bases for EU-US exports, we saw a decrease in aggregate US dollars for 2020 compared to 2019; however, the following year, 2021, the aggregate US dollar amount was more extensive than that of 2019 and followed the sharp trajectory we had hypothesized. In 2020, COVID-19 spread throughout the world, taking everybody by storm. Economies were shut down, people lost their jobs, companies collapsed, and lapses in supply and demand led to a lower aggregate amount of money people spent on said product groups. Although Covid continued, it slowed down, and our economies, for the most part, were able to rebound. Cash handouts from the government encouraged consumption, and demand rose. Supply soon followed as more people were able to return to work as the risk of getting COVID-19 was lower, and so we observed the recovery in all three of the exported product groups graphs in 2021.
Another caveat we ran into was with one of our data sets derived from EuroStat, with HS6 codes for EU exports listed products in aggregate product groups; for example, food could be meat, vegetables, etc.  However, we do not know the specific products in each product category. Therefore, we do not know which products inside the product group were affected by tariffs in 2019 and which of the others were not, changing the aggregate of said category. This issue makes it difficult to get a good picture of the impact of tariffs on EU-US exports as we are accumulating the products that were not affected by tariffs. In future works, finding data that has product based data instead of product group aggregate data would make the research more potent, and possibly lead to an even more extreme outcome. 
The United States confronts a multifaceted challenge as China rapidly ascends to the status of a near-peer competitor across political, economic, military, and technological spheres. Navigating this evolving strategic competition necessitates a nuanced approach that recognizes the undeniable magnitude of China's rise without resorting to a confrontational stance reminiscent of the Cold War. Embracing economic integration and interdependence can yield mutual benefits, avoiding excessive decoupling and maintaining a cooperative economic relationship for global stability. Acknowledging China's advancements in key technology platforms requires striking a balance between healthy competition and collaboration, fostering breakthroughs that address shared global challenges. In the realm of military strategy, the United States must engage in prudent planning to ensure regional stability, combining a credible deterrent with diplomatic channels to manage potential conflicts. Actively participating in multilateral forums allows the United States to shape global norms despite ideological differences, emphasizing specific issues over demonization. Strategic alliances and partnerships with like-minded nations become crucial to amplify influence and create a unified front in addressing global challenges. Ultimately, the United States must acknowledge and navigate the inevitability of major power competition while pursuing a pragmatic and collaborative approach that preserves American interests without succumbing to an entrenched Cold War mentality, fostering a more stable and prosperous global order. We focused on product groups, food, textiles, and capital goods because they represent major components of global trade between the United States and the rest of the world. The post-2018 decrease in trade indicates tariff implementation. Notably, the US-China dataset shows a decline, while the US-EU dataset reveals an increase. We selected these datasets because they help us comprehend a broader shift in trends. Despite focusing on different sectors, they all point towards similar changes, emphasizing the interconnected impact of tariffs on international commerce. Because these product groups encompass a wide variety of exported products, they allow us to assume they convey a similar pattern for European Union exports to the United States as a whole increasing after tariffs and show the effect United States tariffs had on China's exports to the United States as a whole.

\end{itemize}

\section{Conclusion}
\label{sec:conclusion}

Re-state (in different words) what you did and what you learned. If your discussion would be short, you can add the discussion after your summary.

\newpage
\section*{Bibliography}
\singlespacing
\setlength\bibsep{0pt}


\newpage
\section*{Appendix A. Placeholder} \label{sec:appendixa}
\addcontentsline{toc}{section}{Appendix A}

\end{document}


